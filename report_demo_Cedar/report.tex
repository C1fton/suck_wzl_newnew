% !TeX program = xelatex
\documentclass{zjureport}
\special{dvipdfmx:config z 0} % 取消PDF压缩,加快速度,最终版本生成的时候最好把这句话注释掉

% =============================================
% Part 0 Edit the info
% =============================================

\major{电子科学与技术}
\name{韩寒、谌梓轩}
\partner{}
\title{本科实验报告}
\stuid{Group 6}
\college{信息与电子工程学院}
\date{\today}
\lab{东四221}
\course{电子电路设计实验2}
\instructor{王子立}
\grades{}
\expname{电设2课程设计报告}
\exptype{}


\begin{document}
	% =============================================
	% Part 1 Header
	% =============================================
	\makecover
	
	\makecontent
	
	%\makeheader
	% =============================================
	% Part 2 Main document
	% =============================================
	
	
	\section{ECG信号检测调理电路}
	
	\subsection{电路装配调试步骤}
	
	<1>前期PCB版图设计。在实际进行ECG心电信号测试之前,我们先对心电信号处理电路进行原理图设计,并生成PCB图,在仿真过程中,保证其能够正常工作并达到预期效果。
	
	<2>demo板焊接。根据元件装配表,依次进行元件的焊接,并在最后插上运算放大器和芯片。
	
	<3>进行低通滤波部分的测试。将示波器的传输线勾在TP2的针脚和GND上,设置波形发生器从20Hz到140Hz每隔20Hz依次增大,观察示波器上的波形变化,并记录峰峰值。
	
	<4>进行高通滤波部分的测试。将示波器的传输线勾在TP4的针脚和GND上,设置波形发生器从20mHz到80mHz每隔20mHz依次增大,观察示波器上的波形变化,并记录峰峰值。
	
	<5>进行陷波部分的测试。将示波器的传输线勾在TP9和GND上,设置波形发生器从20Hz到140Hz每隔20Hz依次增大,观察示波器波形,记录峰峰值。
	
	<6>进行总体放大性能的测试。将示波器的传输线勾在TP11和GND上,设置波形发生器从20Hz到140Hz每隔10Hz依次增大,观察示波器波形,记录峰峰值。
	
	\subsection{电路测试数据分析}
	
	\subsubsection{低通滤波部分}
	
	在测试低通滤波部分的效果时,我们主要测试20Hz到140Hz部分,部分测试图如图\ref{低通滤波20Hz}、\ref{低通滤波60Hz}、\ref{低通滤波100Hz}、\ref{低通滤波140Hz}所示,测量结果如表\ref{低通滤波20Hz到140Hz测量结果}所示。
	
	\begin{figure}[h]
		\centering
		\begin{minipage}[t]{0.49\linewidth}%%%%%%%%%note2
			\includegraphics[width=0.9\linewidth]{低通滤波20Hz}%%%%%%%%%note3
			\caption{低通滤波20Hz}
			\label{低通滤波20Hz}
		\end{minipage}%
		\begin{minipage}[t]{0.49\linewidth}
			\includegraphics[width=0.9\linewidth]{低通滤波60Hz}
			\caption{低通滤波60Hz}
			\label{低通滤波60Hz}
		\end{minipage}
	
		\begin{minipage}[t]{0.49\linewidth}%%%%%%%%%note2
			\includegraphics[width=0.9\linewidth]{低通滤波100Hz}%%%%%%%%%note3
			\caption{低通滤波100Hz}
			\label{低通滤波100Hz}
		\end{minipage}%
		\begin{minipage}[t]{0.49\linewidth}%%%%%%%%%note2
			\includegraphics[width=0.9\linewidth]{低通滤波140Hz}%%%%%%%%%note3
			\caption{低通滤波140Hz}
			\label{低通滤波140Hz}
		\end{minipage}%
	\end{figure}

	\begin{table}[htbp]
		\centering
		\begin{tabular}{ c|c|c|c|c|c|c|c p{1.5cm}|}
			\hline
			频率(Hz) & 20 & 40 & 60 & 80 & 100 & 120 & 140 \\
			\hline
			峰峰值(V)  & 1.380 & 1.328 & 1.248 & 1.040 & 0.776 & 0.528 & 0.336 \\
			\hline
		\end{tabular}
		\caption{低通滤波20Hz到140Hz测量结果}\label{低通滤波20Hz到140Hz测量结果}
	\end{table}

	我们将表中的测试结果绘制成曲线图,能够更好地看出低通滤波的效果,曲线图如图\ref{低通滤波}所示。从曲线图中,我们可以看出该低通滤波器是可以正常工作、达到低通滤波的效果,并且当其峰峰值下降至-3dB时,频率为84Hz左右,所以该低通滤波器的效果还是不错的,带外的抑制效果也能达到要求。
	
	\begin{figure}[h]
		\centering%使该部分内容居中
		\includegraphics[width=6in]{低通滤波}
		\caption{低通滤波}%命名并编号
		\label{低通滤波}%设置标签
	\end{figure}

	
	\subsubsection{高通滤波部分}

	\begin{figure}[H]
		\centering
		\begin{minipage}[t]{0.4\linewidth}%%%%%%%%%note2
			\includegraphics[width=0.9\linewidth]{高通滤波20mHz}%%%%%%%%%note3
			\caption{高通滤波20mHz}
			\label{高通滤波20mHz}
		\end{minipage}%
		\begin{minipage}[t]{0.4\linewidth}
			\includegraphics[width=0.9\linewidth]{高通滤波40mHz}
			\caption{高通滤波40mHz}
			\label{高通滤波40mHz}
		\end{minipage}
		
		\begin{minipage}[t]{0.4\linewidth}%%%%%%%%%note2
			\includegraphics[width=0.9\linewidth]{高通滤波60mHz}%%%%%%%%%note3
			\caption{高通滤波60mHz}
			\label{高通滤波60mHz}
		\end{minipage}%
		\begin{minipage}[t]{0.4\linewidth}%%%%%%%%%note2
			\includegraphics[width=0.9\linewidth]{高通滤波80mHz}%%%%%%%%%note3
			\caption{高通滤波80mHz}
			\label{高通滤波80mHz}
		\end{minipage}%
	\end{figure}

	在测试高通滤波部分时,主要针对20mHz到80mHz部分,测试图如图\ref{高通滤波20mHz}、\ref{高通滤波40mHz}、\ref{高通滤波60mHz}、\ref{高通滤波80mHz}所示,测试的数据如表\ref{高通滤波20mHz到100mHz测量结果}所示。将测试数据绘制成图像,能得到如图\ref{高通滤波}所示的曲线图。
	
	\begin{table}[htbp]
		\centering
		\begin{tabular}{ c|c|c|c|c p{1.5cm}}
			\hline
			频率(mHz) & 20 & 40 & 60 & 80  \\
			\hline
			峰峰值(V)  & 0.672 & 3.840 & 4.560 & 4.640 \\
			\hline
		\end{tabular}
		\caption{高通滤波20mHz到100mHz测量结果}\label{高通滤波20mHz到100mHz测量结果}
	\end{table}
	
	\begin{figure}[h]
		\centering%使该部分内容居中
		\includegraphics[width=6in]{高通滤波}
		\caption{高通滤波}%命名并编号
		\label{高通滤波}%设置标签
	\end{figure}

	从图中,我们能看出,高通滤波器能够正常工作并起到滤波的效果,当其峰峰值下降为-3dB时,频率约为35mHz,基本满足需求,同时带外抑制的效果也比较好。
	
	\subsubsection{陷波部分}
	
	在陷波部分,我们为得到陷波的频率,设置了从50Hz到35Hz的测试范围,如图\ref{陷波最低点}所示是测试时峰峰值最低点的图像。最终得到的测试数据如表\ref{陷波测试结果}所示。根据表中测试数据,我们绘制出陷波曲线,如图所示,能够更好地看出陷波地效果。
	
	\begin{figure}[H]
		\centering%使该部分内容居中
		\includegraphics[width=3in]{陷波最低点}
		\caption{陷波最低点}%命名并编号
		\label{陷波最低点}%设置标签
	\end{figure}
	
	\begin{table}[htbp]
		\centering
		\begin{tabular}{ c|c|c|c|c|c|c|c|c p{1.5cm}}
			\hline
			频率(Hz) & 50 & 49 & 48 & 47 & 46 & 45 & 44 & 43  \\
			\hline
			峰峰值(V)  & 3.040 & 2.960 & 2.840 & 2.720 & 2.600 & 2.480 & 2.300 & 2.140 \\
			\hline
		\end{tabular}
	\end{table}

	\begin{table}[H]
		\centering
		\begin{tabular}{ c|c|c|c|c|c|c|c p{1.5cm}}
			\hline
			 42 & 41 & 40 & 39 & 38 & 37 & 36 & 35  \\
			\hline
			 1.980 & 1.800 & 1.640 & 1.560 & 1.460 & 1.520 & 1.640 & 1.880 \\
			\hline
		\end{tabular}
		\caption{陷波测试结果}\label{陷波测试结果}
	\end{table}

	\begin{figure}[H]
		\centering%使该部分内容居中
		\includegraphics[width=6.5in]{陷波测试结果}
		\caption{陷波测试结果}%命名并编号
		\label{陷波测试结果图}%设置标签
	\end{figure}

	从图中,我们可以看出,陷波的中心位置为38Hz,更细致应该是在37到38Hz之间,而我们希望的陷波位置应该为40到50Hz之间,且该滤波测试结果中,陷波地带宽比较大,合格的陷波器的陷波位置应该是比较窄的,因此,陷波部分地效果并不理想。效果不理想的原因,最主要的应该是焊接使用的电容的电容值偏离指定值,导致陷波位置发生了偏移。
	
	\subsubsection{总体放大性能}
	
	在心电信号处理电路各个单独部分测试完成后,我们进行电路总体放大性能的测试,设置了20Hz到140Hz的测试范围,其测试结果如表\ref{总体性能测试结果}所示。根据表中数据,绘制得到如图\ref{总体放大性能}所示的曲线。
	
	\begin{table}[htbp]
		\centering
		\begin{tabular}{ c|c|c|c|c|c|c|c|c|c|c|c|c|c p{1.5cm}}
			\hline
			频率(Hz) & 20 & 30 & 40 &50 & 60 &70 & 80 &90 & 100 & 110 & 120 &130 & 140 \\
			\hline
			峰峰值(V)  & 2.580 & 2.120 & 0.900 & 1.780 & 2.160 & 2.200 & 2.060 & 1.840 & 1.580 & 1.380 & 1.140 & 0.960 & 0.752 \\
			\hline
		\end{tabular}
		\caption{总体性能测试结果}\label{总体性能测试结果}
	\end{table}

	\begin{figure}[H]
		\centering%使该部分内容居中
		\includegraphics[width=6.5in]{总体放大性能}
		\caption{总体放大性能}%命名并编号
		\label{总体放大性能}%设置标签
	\end{figure}

	从图中可以看出,在40Hz左右发生陷波,在30Hz之前和45到101Hz之间增益较高,其余位置的抑制都较为明显,基本符合我们设计的初衷,能够起到总体处理和放大心电信号的作用。
	
	\subsubsection{人体测试}
	
	在测试完ECG信号检测调理电路的各类参数性能之后,我们进行人体测试,观察该电路实际能否起到处理心电信号,最终输出合格波形的作用。测试结果如图所示。
	
	\begin{figure}[H]
		\centering%使该部分内容居中
		\includegraphics[width=4in]{ECG信号检测调理电路_人体测试}
		\caption{ECG信号检测调理电路-人体测试}%命名并编号
		\label{ECG信号检测调理电路_人体测试}%设置标签
	\end{figure}

	在示波器上,我们已经可以看出心电信号的波形,但同时,也发现其中混叠的杂波较多,心跳与心跳之间干扰明显,心电信号的增益还不够大,容易和杂波混在一起。
	
	\section{ECG心电监测系统}
	
	\subsection{总体目标}
	
	
	
	\subsubsection{系统框图及功能说明}
	
	\subsection{硬件设计与调试}
	
	\subsubsection{原理图设计}
	
	\subsubsection{PCB设计}
	
	\subsubsection{硬件调试}
	
	(1)共模抑制比
	
	(2)差模幅频特性
	
	(3)脉冲信号波形
	
	(4)生物电信号波形
	
	(5)发射信号功率
	
	(6)螺旋天线制作与测试
	
	(7)拉距实验及结果分析(传输距离与功率、频率的关系,误码率等)
	
	\subsubsection{硬件电路整体性能分析}
	
	\subsection{软件设计与调试}
	
	\subsubsection{软件工作流程框图}
	
	\subsubsection{软件功能设计}
	
	(1)时序设计,如定时器,AD采样时间规划   
	    
	(2)AD功能                   
	              
	(3)串口功能              
	                
	(4)nRF905功能            
	            
	(5)其他
	
	
	\subsubsection{软件调试}
	
	(1)调试环境和方法
	
	(2)波形采样,串口显示   
	               
	(3)模块之间数据通讯      
	
	\section{总结}
	
	\subsection{系统装调过程中遇到的问题和解决情况}
	
	\subsection{心得体会}
	
	\subsubsection{韩寒}
	
	在这个学期的电子电路设计实验中,我学到了很多有用的新知识。
	
	首先,我们完成了上个学期绘制的两块PCB的装配与调试,在调试过程中,我发现人体信号的检测很困难,在老师的指导下,我了解到在电路中接入右脚屏蔽端的重要性。同时,在进行两块PCB的小信号性能测试过程中,我又对信号发生器、模拟心电信号发生器等设备的使用掌握的更加熟练。
	
	在软件设计部分,我在之前的学习中已经接触过C51及Keil的开发使用,而一学期的设计下来,我对使用Keil与Debugger联调、以及利用串口调试助手进行检验等技能掌握的更加熟练。同时老师提供的$Silicon \quad Lab$以及$Config2$软件也提高了我进行开发的效率。
	
	在软件设计完成后,我们小组反复进行调试,最终实现了按照100hz发送心电数据、借助按键实现射频工作频率及发射功率的调节、以及利用天线进行远距离射频传输等功能。实验课最后,我们还使用$MATLAB$开发了一个读取串口数据并实时绘图、计算心率的小程序。这些工作都很大程度上提高了我的编程水平以及调试能力,对我以后的学习生活有很大的帮助。
	
	总而言之,一个学期的电设实验下来,我对ECG监测、调试等内容都有了更加形象、深刻的认识,也大大提高了硬件调试、软件编程等方面的能力。通过实践结合理论的方式,我对之前电子电路基础课程中的学习内容也有了更加深入的理解,我相信我能将从电设实验课程上学会的知识充分地运用到我之后的学习之中,成为一名更加优秀的信电学子!
	
	\subsubsection{谌梓轩}
	
	\subsection{组员分工}
	
	
	
\end{document}