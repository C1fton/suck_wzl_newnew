% !TeX program = xelatex
\documentclass{zjureport}
\special{dvipdfmx:config z 0} % 取消PDF压缩,加快速度,最终版本生成的时候最好把这句话注释掉

% =============================================
% Part 0 Edit the info
% =============================================

\major{电子科学与技术}
\name{雪松}
\partner{}
\title{本科实验报告}
\stuid{3210100000}
\college{信息与电子工程学院}
\date{2022-11-10}
\lab{东四216}
\course{电子工程训练(甲)}
\instructor{}
\grades{}
\expname{智能插座实验}
\exptype{}


\begin{document}
% =============================================
% Part 1 Header
% =============================================
\makecover

% \makecontent

\makeheader
% =============================================
% Part 2 Main document
% =============================================

% \begin{figure}[H] %H为当前位置,!htb为忽略美学标准,htbp为浮动图形
%     \centering %图片居中
%     \includegraphics[width=0.8\textwidth]{fig1-SR830structure.jpg}
%     \caption{锁相放大器SR830结构图} %最终文档中希望显示的图片标题
% \end{figure}


\begin{equation}
    \left\{\begin{array}{l}
    \tilde{V}_{r 1}(\omega)=\pi A_r\left[e^{-i \varphi_r} \delta\left(\omega+\omega_s\right)+e^{i \varphi_r} \delta\left(\omega-\omega_s\right)\right] \\
    \tilde{V}_{r 2}(\omega)=i \pi A_r\left[e^{-i \varphi_r} \delta\left(\omega+\omega_s\right)-e^{i \varphi_r} \delta\left(\omega-\omega_s\right)\right]
    \end{array}\right.
\end{equation}


\begin{table}[H]
    \caption{测温电路标定调试}
    \centering
    \begin{tabular}{L{8cm}C{3cm}}
      \toprule
      项目                                       & 测量值         \\
      \midrule
      当前实际室温(摄氏度):                   & 18\textcelsius \\
      \\[2pt]
      经你完成调试后测试程序显示温度(摄氏度): & 22\textcelsius \\
      \\[2pt]
      是存在严重的元器件离散性问题?(是/否)    & 否             \\
      \bottomrule
    \end{tabular}
\end{table}

\section{实验目的}


\section{实验任务与要求}


\section{实验方案设计与实验参数计算}


\section{主要仪器设备}


\section{实验步骤、调试过程和数据记录}


\section{实验结果和分析处理}


\section{讨论、心得}


\end{document}