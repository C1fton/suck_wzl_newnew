% !TeX program = xelatex
\documentclass{zjureport}
\special{dvipdfmx:config z 0} % 取消PDF压缩,加快速度,最终版本生成的时候最好把这句话注释掉

% =============================================
% Part 0 Edit the info
% =============================================

\major{电子科学与技术}
\name{韩寒、谌梓轩}
\partner{}
\title{本科实验报告}
\stuid{Group 6}
\college{信息与电子工程学院}
\date{\today}
\lab{东四221}
\course{电子电路设计实验2}
\instructor{王子立}
\grades{}
\expname{电设2课程设计报告}
\exptype{}


\begin{document}
	% =============================================
	% Part 1 Header
	% =============================================
	\makecover
	
	\makecontent
	
	%\makeheader
	% =============================================
	% Part 2 Main document
	% =============================================
	
	
	\section{ECG信号检测调理电路}
	
	\subsection{电路装配调试步骤}
	
	<1>前期PCB版图设计。在实际进行ECG心电信号测试之前,我们先对心电信号处理电路进行原理图设计,并生成PCB图,在仿真过程中,保证其能够正常工作并达到预期效果。
	
	\subsection{电路测试数据分析}
	
	\subsubsection{低通滤波部分}
	
	\subsubsection{高通滤波部分}
	
	\subsubsection{陷波部分}
	
	\section{ECG心电监测系统}
	
	\subsection{总体目标}
	
	\subsubsection{系统框图及功能说明}
	
	\subsection{硬件设计与调试}
	
	\subsubsection{原理图设计}
	
	\subsubsection{PCB设计}
	
	\subsubsection{硬件调试}
	
	\subsubsection{硬件电路整体性能分析}
	
	\subsection{软件设计与调试}
	
	\subsubsection{软件工作流程框图}
	
	\subsubsection{软件功能设计}
	
	\subsubsection{软件调试}
	
	\section{总结}
	
	\subsection{韩寒:}
	
	\subsection{谌梓轩:}
	
	
\end{document}